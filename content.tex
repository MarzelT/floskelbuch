\chapter{Auflistungen}
\label{sec:auflistungen}

In diesem Kapitel sollen Beispiele von Auflistungen aufgeführt werden. Augenmerk liegt hier auf der Einführung einer Auflistung sowie der grundsätzlichen Formulierung eines Listenpunkts. 

Als weitere Punkte für zukünftige Entwicklungen am resonatorintern frequenzverdoppelten \tisa-Laser sind zu nennen \cite{Naubereit.2012}: 

\begin{itemize}
	\item Messungen zur Linienbreite des Systems [\dots] sind wünschenswert und müssen durchgeführt werden. 
	\item Durch eine bessere Beschichtung des Verdopplungskristalls können Verluste durch Reflexionen vermieden werden, was erprobt werden sollte. 
	\item \dots
	\item Als letzter Aspekt kann eruiert werden, in wieweit eine unkritische Phasenanpassung über Temperaturveränderung lohnend ist. Der mit einem Kristall abzudeckende Wellenlängenbereich ist sicherlich kleiner. Allerdings kann [\dots] durchgeführt werden. 
\end{itemize}

Um einen funktionierenden Laserprozess zu erzeugen \cite{Naubereit.2014} und aufrechtzuerhalten sind im Allgemeinen drei Bausteine essentiell: 

\begin{itemize}
	\item \textbf{Aktives Medium} bezeichnet ein Material, welches potentiell das Vermögen hat den Laserprozess auszuführen. Nach der Beschaffenheit des aktiven Mediums [\dots]. 
	\item \textbf{Energiequelle} wird je nach Art auch als Pumpquelle oder Pumplaser bezeichnet. Je nach Lasertyp gibt es unterschiedliche Möglichkeiten, die nötige Energie zuzuführen [\dots].
	\item{} [\dots].
\end{itemize}

Für einen funktionierenden Laserresonator müssen zwei notwendige Bedingungen erfüllt sein \cite{Naubereit.2014}: 
\begin{enumerate}
	\item \textbf{Schwellenbedingung} Die Verstärkung durch die stimulierte Emission [\dots]. 
	\item \textbf{Resonanzbedingung} In einem Stehwellenresonator werden nur [\dots]. 
\end{enumerate}


\chapter{Abbildungen}
\label{sec:abbildungen}

Hier führe ich ein paar Formulierungen für die Referenzierung von Abbildungen auf. 

\begin{itemize}
	\item Abbildung~1.1 zeigt\dots
	\item \dots wie in Abbildung~1.1 dargestellt. 
	\item In Abbildung~1.1 sind \dots gezeigt. 
	\item \dots sind in Abbildung~1.1 zu sehen. 
	\item In Abbildung~1.1 sind \dots aufgeführt. 
	\item \dots ist in Abbildung~1.1 grafisch dargestellt und zeigt \dots
	\item \dots, dargestellt in Abbildung~1.1, \dots
	\item In Abbildung~1.1 sind \dots veranschaulicht. 
	\item \dots, siehe hierzu Abbildung~1.1.
\end{itemize}